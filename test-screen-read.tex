% 24 Jan 2012  
% 22 Jan 2012  


% Test of package screenread.sty that gives a layout suitable for
%    screen reading.  It is based on ideas in
%  B. Veytsman amd M. Ware, Tugboat 32 (2011) 261.


\documentclass{article}
\usepackage{lipsum}
\usepackage{url}

\usepackage{screenread}

%-----------

\providecommand{\eqdef}{\stackrel{\textrm{def}}{=}}

%-----------

\newif\ifscreen
\makeatletter
  \@ifpackageloaded{screenread}{\screentrue}{\screenfalse}
\makeatother

\title{Test of package \texttt{screenread.sty}
          \\
          \ifscreen
            (Document suitable for screen reading)
          \else
            (Document suitable for printing)
          \fi
   }
\author{John Collins}
\date{24 January 2012}


%===========================================

\begin {document}
\maketitle

\begin{abstract}
This document exercises the package \texttt{screenread.sty}.
First, it stress tests footnote handling with enormous footnotes, and
then shows the content of a handout from a physics course.
\end{abstract}

\section{Stress test for footnotes}

Here is some text:
\begin{quotation}
  \lipsum[6-20]
\end{quotation}
It should also appear in the following extremely long footnotes
\footnote{\lipsum[6-10]}
and
\footnote{\lipsum[11-20]}

\lipsum[1]

\pagebreak
After a manual page break
\lipsum[2]


%============================================================
\pageBreakSection{Summary of work on standard quantization}

\url{http://wikipedia.org/physics}

There are standard rules of canonical quantization to convert systems
described by a Lagrangian formalism to quantum mechanics.  The rules
are \emph{stated} in standard textbooks.  However, it is important to
know not only what the rules are but also \emph{why} they are what
they are, in order to deal with new situations, for example
Lagrangians linear in time derivatives of the generalized coordinates,
and, importantly for many people here, gravity.

The standard rules were set up in the founding papers of quantum
mechanics, in the context of ordinary systems in Newtonian mechanics,
with a Lagrangian quadratic in time derivatives.  If you are
interested in the rules for getting a quantum theory from a
Lagrangian, the founding papers of quantum mechanics 
are important because they do
not merely state the rules; they also so strongly motivate the canonical
quantization rules that the rules can almost be regarded as proved from
more elementary considerations, provided that you stay in the context
of the systems to which they are applied.  The key papers are as
follows:
\begin{itemize}
\item Heisenberg \cite{Heisenberg} started out the subject, and
  inspired the next papers.
\item Born and Jordan \cite{Born.Jordan} worked out the rules.  Their
  paper includes a treatment of the electromagnetic field.
\item Dirac \cite{Dirac} independently worked out the rules.
\item Born, Heisenberg, and Jordan \cite{Born.Heisenberg.Jordan} gave
  a complete formulation.
\end{itemize}

All the above papers (translated into English when the originals are
in German) are to be found in van der Waerden's book \cite{Waerden}.
But note that the translation of the Born and Jordan paper omits the
section on quantum fields.  I have made a translation of the missing
section available at
\url{http://www.phys.psu.edu/~collins/564/born_jordan.pdf}.

%=======

\pageBreakSection{Another section}
with some words for efekt

%============================================================
\pageBreakSection{First-order Lagrangians}

\subsection{Literature}
As for Lagrangians linear in time derivatives, key papers relevant to
the methods we will use in this course are as follows:
\begin{itemize}
\item Floreanini and Jackiw \cite{Floreanini.Jackiw} stated the
  rules. 
\item Costa and Girotti \cite{Costa.Girotti} argued that the previous
  paper should have used constrained quantization and showed how to do
  it.
\item Faddeev and Jackiw \cite{Faddeev.Jackiw} rebut Costa and
  Girotti, with a more detailed explanation of why you do not need
  Dirac's constrained quantization to deal with this case.
\end{itemize}

Another method that is frequently quoted in this context, but that is
considerably more complicated is Dirac's method of constrained
quantization, for which a standard reference is in a little book by
Dirac \cite{Dirac.constrained}.  For a detailed treatment of
constrained quantization, in the context of $(2+1)$ and $(3+1)$
gravity, see \cite{Romano}.  Although the main aim of this reference
concerns applications to general relativity, it also has a simple
example treated at its Eq.\ (2.11).

In this course I will use the Floreanini-Faddeev-Jackiw method,
because it is very much simpler and easier.




\subsection{Formulation}
The rules are stated by Floreanini and Jackiw \cite{Floreanini.Jackiw}
and in more detail by Faddeev and Jackiw \cite{Faddeev.Jackiw}.

\paragraph{Specification of system}
We work with a Lagrangian that is of \emph{first} order in time
derivatives, rather than second order, which is the usual case.  The
coordinates are even in number, collectively denoted as $\underline{q}
= (q^1, q^2, \dots, q^{2N})$, and the Lagrangian is of the form
\begin{equation}
\label{eq.L1}
  L = \frac{1}{2} \sum_{j,k} q^j f_{jk} \dot{q}^k - F(\underline{q}).
\end{equation}
Here $f$ is an invertible \emph{anti}symmetric\footnote{Note that by
  adding a time derivative, other forms may be obtained, e.g.,
  $q^1\dot{q}^2$ instead of $\frac{1}{2}(q^1\dot{q}^2-q^2\dot{q}^1)$.
  Such changes do not affect the equations of motion or the
  Hamiltonian.  But the form (\ref{eq.L1}) is the most symmetric, so
  we will take it as the canonical one.} numerical matrix,
$f_{jk}=-f_{kj}$, and $F(\underline{q})$ is some function of the
coordinates.  Thus $F$ appears to be like the potential in the usual
second-order Lagrangians of classical mechanics etc.

\paragraph{Euler-Lagrange equations and Hamiltonian}
There is \emph{exactly} no change from the usual case
either in the form of the Euler-Lagrange equations,
\begin{equation}
  \label{eq:E-L}
  0 = \frac{d}{dt} \frac{\partial L}{\partial \dot{q}^j} - \frac{\partial L}{\partial q^j},
\end{equation}
or in the definition of the Hamiltonian,
\begin{equation}
  \label{eq:H}
  H(\underline{q})
  \eqdef \sum_j \frac{\partial L}{\partial \dot{q}^j} \dot{q}^j - L
  = F(\underline{q}).
\end{equation}
Thus $H$ is the Noether charge for time-translation invariance.  

\paragraph{Equal-time canonical commutation relations}
The equal-time canonical commutation relations (ETCCR) between the
$q^j$s are to be such that the ordinary Heisenberg equation of motion
in the quantum theory,
\begin{equation}
  \label{eq:e.of.m}
  i \hbar \frac{dq^j(t)}{dt} = [q^j(t), H],
\end{equation}
agree with the Euler-Lagrange equations Eq.\ (\ref{eq:E-L}).  We do
this by imposing
\begin{equation}
  \label{eq:ETCCR}
  [q^j(t), q^k(t)] = i \hbar (f^{-1})^{jk} .
\end{equation}
Here $f^{-1}$ is defined to obey $\sum_k (f^{-1})^{jk} f_{kl} = \delta^j_l$.
These commutation relations comprise the only changes compared with
the usual case.  \textbf{You should verify that} the Euler-Lagrange
equations of motion and the Heisenberg equations of motion agree, up
to operator ordering issues.  It is particularly important to check
the sign of the ETCCR.


\paragraph{Canonical momenta}
The above formulation has bypassed a definition of canonical momenta.
We can obtain them by modifying the usual definition to
\begin{equation}
  \label{eq:mom}
  p_j \eqdef 2 \frac{\partial L}{\partial \dot{q}^j} = \sum_k q^k f_{kj}.
\end{equation}
Then we use just the usual formula for the ETCCR to
\begin{equation}
  \label{eq:ETCCR.qp}
  [q^k(t), p_j(t)] = i \hbar \delta^k_j
\end{equation}
in order to reproduce Eq.\ (\ref{eq:ETCCR}).  We must also ignore the
usual formulae for the ETCCR for the $p_j$s with themselves and for
the $q^j$s with themselves; the relevant formulae are derivable from
Eq.\ (\ref{eq:ETCCR}) since the $p_j$s are determined by the $q^j$s.

An alternative \emph{choice} is to define the $p_j$s without the
factor of two on the right of Eq.\ (\ref{eq:mom}); then a factor of
$1/2$ is inserted on the right of Eq.\ (\ref{eq:ETCCR.qp}).

But these are both mere definitions.  Since the momenta are uniquely
determined by the coordinates, it is probably best to avoid the
concept of canonical momenta for these systems, and just to impose
Eq.\ (\ref{eq:ETCCR}) directly on the $q^j$s.  In the usual case, we
need both the $q^j$s and the $p_j$s to specify the state of a
classical system at a particular time.  In the case of a first-order
Lagrangian only the coordinates are needed.

\paragraph{Mixed first- and second-order Lagrangians}
It is left as an exercise to work out the formalism for a Lagrangian
which has only first-order terms for some of the velocities
$\dot{q}^j$, and second-order terms for the others.


\paragraph{Relation to Dirac constrained quantization}
Seriously interested students might find it interesting to understand
the relation between the above formalism and the Dirac formalism for
constrained quantization \cite{Dirac.constrained,Henneaux.Teitelboim}.


%============================================================

\begin{thebibliography}{99}

\bibitem{Heisenberg} W. Heisenberg, ``Quantum-theoretical
  re-interpretation of kinematic and mechanical relations'', Zeit.\
  f.\ Phys.\ \textbf{33} 879--893 (1925).

\bibitem{Born.Jordan} M. Born and P. Jordan, ``On quantum mechanics'',
  Zeit.\ f.\ Phys.\ \textbf{34}, 858--888 (1925).

\bibitem{Dirac} P.A.M. Dirac, ``The fundamental equations of quantum
  mechanics'', Proc.\ Roy.\ Soc.\ A \textbf{109}, 642--653 (1926).

\bibitem{Born.Heisenberg.Jordan} M. Born, W. Heisenberg and P. Jordan,
  ``On quantum mechanics II'', Zeit.\ f.\ Phys.\ \textbf{35}, 557--615
  (1926).

\bibitem{Waerden} B.L. van der Waerden, ``Sources of quantum
  mechanics'' (Dover, 1968) QC174.1.W3 1968 in the PSU library.

\bibitem{Dirac.constrained} P.A.M. Dirac ``Lectures on Quantum
  Mechanics'' (Yeshiva University, New York, 1964).

\bibitem{Floreanini.Jackiw} R. Floreanini and R. Jackiw, Phys.\ Rev.\
  Lett.\ \textbf{59}, 1873 (1987).

\bibitem{Costa.Girotti} M.E.V. Costa and H.O. Girotti, Phys.\ Rev.\
  Lett.\ \textbf{60}, 1771 (1988).

\bibitem{Faddeev.Jackiw} L.D. Faddeev and R. Jackiw, Phys.\ Rev.\
  Lett.\ \textbf{60}, 1692 (1988).

\bibitem{Romano} J.D. Romano, ``Geometrodynamics vs.\ connection
  dynamics'', Gen.\ Rel.\ Grav.\ {\bf 25}, 759 (1993)
  [arXiv:gr-qc/9303032]

\bibitem{Henneaux.Teitelboim} M. Henneaux and C. Teitelboim,
  ``Quantization of gauge systems'' (Princeton University Press,
  1992).
  \footnote{A test footnote, to check that it is correctly treated at
    the end of the document.}

\end{thebibliography}

\end {document}
